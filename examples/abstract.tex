%%%%%%%%%%%%%%%%%%%%%%%%%%%%%% -*- Mode: Latex -*- %%%%%%%%%%%%%%%%%%%%%%%%%%%%
%% uhtest-abstract.tex -- 
%% Author          : Robert Brewer
%% Created On      : Fri Oct  2 16:30:18 1998
%% Last Modified By: Robert Brewer
%% Last Modified On: Fri Oct  2 16:30:25 1998
%% RCS: $Id: uhtest-abstract.tex,v 1.1 1998/10/06 02:06:30 rbrewer Exp $
%%%%%%%%%%%%%%%%%%%%%%%%%%%%%%%%%%%%%%%%%%%%%%%%%%%%%%%%%%%%%%%%%%%%%%%%%%%%%%%
%%   Copyright (C) 1998 Robert Brewer
%%%%%%%%%%%%%%%%%%%%%%%%%%%%%%%%%%%%%%%%%%%%%%%%%%%%%%%%%%%%%%%%%%%%%%%%%%%%%%%
%% 

\begin{abstract}
There are several key components of health, which can be monitored through means of body composition
and shape analysis. Metrics such as fat mass index and fat free mass index are simple features of the humany body that can aid our understanding in their overall health. How these are measured have been extensively studied throughout the years. Means such as DXA and bioimpedance can accurately measure body fat. While more recently, 3d imaging has proven to be able to match the performance of its predecessors whilst providing certain benefits such as lower cost. The Shepherd Lab has demonstrated such findings by showing similar results between DXA and 3d imaging. The project is ongoing and this thesisis is the culmination of about 6 months relating to 3d imaging methods and testing for applications in space. It provides an overview of my work and the starting point of much future effort.
\end{abstract}
