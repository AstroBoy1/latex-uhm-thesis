%%%%%%%%%%%%%%%%%%%%%%%%%%%%%% -*- Mode: Latex -*- %%%%%%%%%%%%%%%%%%%%%%%%%%%%
%% uhtest-abstract.tex -- 
%% Author          : Robert Brewer
%% Created On      : Fri Oct  2 16:30:18 1998
%% Last Modified By: Robert Brewer
%% Last Modified On: Fri Oct  2 16:30:25 1998
%% RCS: $Id: uhtest-abstract.tex,v 1.1 1998/10/06 02:06:30 rbrewer Exp $
%%%%%%%%%%%%%%%%%%%%%%%%%%%%%%%%%%%%%%%%%%%%%%%%%%%%%%%%%%%%%%%%%%%%%%%%%%%%%%%
%%   Copyright (C) 1998 Robert Brewer
%%%%%%%%%%%%%%%%%%%%%%%%%%%%%%%%%%%%%%%%%%%%%%%%%%%%%%%%%%%%%%%%%%%%%%%%%%%%%%%
%% 

\begin{abstract}
There are several key components of health, which can be monitored through means of body composition
and shape analysis. Metrics such as fat mass index and fat free mass index are simple features of the human body that can aid our understanding in their overall health. How these are measured have been extensively studied throughout the years. Means such as DXA and bioimpedance can accurately measure body fat. While more recently, 3D optical imaging has been shown to be a promising alternative whilst providing benefits such as lower cost. The Shepherd Lab has demonstrated such findings by showing 3D optical imaging to track well with DXA. The project is ongoing and this thesis is the culmination of my progress relating to 3D optical imaging methods and testing for applications in microgravity. It provides an overview of my work and the starting point of much future effort.

Through experiments, I optimized the equipment to maximize the performance of sensors and our imaging system for use in a parabolic flight setup utilizing microgravity simulation protocols to improve the measurement of health metrics through both performance and speed. I showed the importance of settings such as frame rate, resolution, and noise and how the specs may look similar between sensors but by recording metrics such as fill rate, we can see large areas of performance differences. With regards to the imaging configuration, I documented the relationship between certain settings and the resulting 3d reconstruction using quantitative metrics in which the distance from the sensor to the object is of high importance. I created a system that can be used on a parabolic flight with microgravity and future testing further out in space. Several models for learning body fat using a graph neural network, extreme gradient boosted decision trees were trained, and I produced a system for the first automated Caesar dataset landmark placement using deep learning which may hopefully greatly accelerate any pipelines that use this for template registration utilizing a robustly trained version.

\end{abstract}
